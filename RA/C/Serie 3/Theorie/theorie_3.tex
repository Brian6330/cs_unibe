\documentclass[a4paper,10pt,headlines=3.2]{scrartcl}


% Basispakete
\usepackage[utf8]{inputenc}
\usepackage{amsmath}
\usepackage[automark]{scrpage2}
\usepackage{graphicx}
\usepackage{amssymb}
\usepackage{tikz}
\usepackage{ifthen}
\usepackage{calc}
\usepackage{tikz}
\usepackage{ngerman}

\usepackage{graphicx}           %Bilder
\usepackage{amsmath}            %Math. Zeichen
\usepackage{pifont}             %Skalierbare Schriftart
\usepackage{array}
\usepackage{epsfig}             %Erweiterte Grafiken
\usepackage{makeidx}            %Stichwortverzeichnis
%\usepackage[pdftex]{color} 

\newcommand{\changefont}[3]{
\fontfamily{#1} \fontseries{#2} \fontshape{#3} \selectfont}

\makeindex

\usepackage[automark]{scrpage2}
\usepackage[nosectionbib]{apacite}               %Zitieren

%\usepackage[colorlinks]{hyperref}%Hyperlinks

\usepackage{lmodern}
\usepackage{scrpage2}           %KOMA-Script
\usepackage{tipa}
\usepackage{qtree}

\usepackage{remreset}			%Fussnoten global
\makeatletter
\@removefromreset{footnote}{chapter}
\makeatother 

\setcounter{tocdepth}{3}

%Kopfzeilen
\pagestyle{scrheadings}         %Seitenstil scrheadings verwenden

%\setlength{\textheight}{24cm}
%\setlength{\textwidth}{16cm}
%\setlength{\topmargin}{-2cm}
%\setlength{\oddsidemargin}{0cm}

% Groesse des Textbereiches in der Seite
\setlength{\textwidth}{16cm}
\setlength{\textheight}{21.3cm}
% Kopf- und Fusszeile, Hoehe und Abstand vom Text
\setlength{\headheight}{15pt}
\setlength{\headsep}{0.8cm}
% Linker Seiteneinzug
\setlength{\oddsidemargin}{2.5cm} \addtolength{\oddsidemargin}{-1in}
\setlength{\evensidemargin}{2.5cm} \addtolength{\evensidemargin}{-1in}
% Andere Groessen ausrechnen (vertikal zentrieren)
\setlength{\footskip}{\headsep}
\addtolength{\footskip}{\headheight}
\setlength{\topmargin}{\paperheight}
\addtolength{\topmargin}{-\textheight}
\addtolength{\topmargin}{-\headheight}
\addtolength{\topmargin}{-\headsep}
\addtolength{\topmargin}{-\footskip}
\addtolength{\topmargin}{-2in}
\addtolength{\topmargin}{-0.5\topmargin}

%Abstand zurücksetzen
\setlength{\headheight}{20pt}

\usepackage{listings} 
\lstset{numbers=left, numberstyle=\tiny, numbersep=5pt} \lstset{language=Java} 
\changefont{cmss}{m}{n}

%Tabellen
\usepackage{booktabs}
\usepackage{multirow}

%Kopfzeilen
\clearscrheadfoot
%\renewcommand{\headheight}{40pt} 
\ihead[]{Rechnerarchitektur \\Frühlingssemester 2012 \\Institut für angewandte
Mathematik} % - linke Kopfzeile 
\ohead[asdasd]{Übung 3, Abgabe 18. April 2012 \\ Adrianus Kleemans
[07-111-693] \\Christa Biberstein [09-104-381]} % - linke Kopfzeile 
\setheadsepline{.4pt} %Separate Linie im Kopf
\cfoot[\pagemark]{\pagemark} %- mittlere Fusszeile 

% Nummerierung
\renewcommand \thesection {\arabic{section}.}
\renewcommand \thesubsection {\arabic{section}.\arabic{subsection}.}
\renewcommand \thesubsubsection {\arabic{section}.\arabic{subsection}.\arabic{subsubsection}.}

\begin{document}
\section{Performance-Berechnungen}
\begin{center}
\begin{tabular}{llll}
\textbf{Op} & \textbf{Freq} & \textbf{$CPI_i$} & \textbf{Freq x $CPI_i$}\\
\texttt{ALU} & 25\% & 5 & 1.25\\
\texttt{LOAD} & 25\% & 10 & 2.5\\
\texttt{STORE} & 25\% & 7.5 & 1.875\\
\texttt{Branch} & 25\% & 7.5 & 1.875\\
 & & & $\Sigma$ = 7.5
\end{tabular}
\end{center}
a) Die Tabelle sieht dann wie folgt aus:
\begin{center}
\begin{tabular}{llll}
\textbf{Op} & \textbf{Freq} & \textbf{$CPI_i$} & \textbf{Freq x $CPI_i$}\\
\texttt{ALU} & 25\% & 5 & 1.25\\
\texttt{LOAD} & 25\% & 6 & 1.5\\
\texttt{STORE} & 25\% & 7.5 & 1.875\\
\texttt{Branch} & 25\% & 7.5 & 1.875\\
 & & & $\Sigma$ = 6.5
\end{tabular}
\end{center}
Die CPU ist $13.3\%$ schneller. \\\\
b)
\begin{center}
\begin{tabular}{llll}
\textbf{Op} & \textbf{Freq} & \textbf{$CPI_i$} & \textbf{Freq x $CPI_i$}\\
\texttt{ALU} & 25\% & 2.5 & 0.625 \\
\texttt{LOAD} & 25\% & 6 & 2.5\\
\texttt{STORE} & 25\% & 7.5 & 1.875\\
\texttt{Branch} & 25\% & 7.5 & 1.875\\
 & & & $\Sigma$ = 6.875
\end{tabular}
\end{center}
Die CPU ist $8.3\%$ schneller. 

\section{Stackverwendung bei Subroutinen}
\begin{itemize}
\item Beim Aufruf von Subroutinen wird Speicherplatz für die lokalen Variablen der Funktion reserviert (``stack frame'', damit verbunden der ``Frame Pointer''). \\
\item Ein weiterer Verwendungszweck besteht darin, dass die Parameter auf dem Stack abgelegt und zwischengespeichert werden, damit sie von der Subroutine weiterverwendet werden können.
\end{itemize}

\section{ALU \& Most Significant Bit}
Die ALU muss für das \textit{most significant bit} deshalb anders aufgebaut sein, damit \texttt{slt} unterstützt werden kann. Das \textit{msb} gibt als einziges Bit das \texttt{Less} weiter (an das \textit{lsb}). Zudem muss der Overflow abgefangen werden, weshalb diese Leitung nicht zum nächst höheren Bit führt (es gibt ja kein höherwertiges Bit).
\newpage

\section{ALU \& \texttt{SLT}}
\begin{align}
A\,slt\,B = \left \{ \begin{matrix} 0...01\text{ if } A < B & \text{ i.e. if } A-B < 0 \\ 0...00\text{ if } A \geq B & \text{ i.e. if } A-B \geq 0 \end{matrix} \right.
\end{align}
Beim \texttt{slt}-Befehl (\textit{set on less than}) wird beim ``Operation''-control der Schalter auf 3 gesetzt. Dies bewirkt, dass beim höchstwertigen Bit das set übertragen wird auf dass less beim tiefstwertigen Bit. Alle Bits ausser dem niederwertigsten haben bei \texttt{less} 0 als Input. \\
Der Trick ist nun dass die Bedingung $A < B$ umformuliert werden kann zu $A-B < 0$. Bei dieser Operation kann einfach das höchstwertige Bit betrachtet werden, welches angibt, op $A-B$ negativ ist. Ist dies der Fall, muss $A < B$ gelten und das Bit wird übertragen.

\section{Pop und push}
\begin{itemize}
\item \texttt{pop}: Der Wert wird geladen und z.B. in Register \texttt{\$r3} gespeichert. Danach wird zum (Stack-)Pointer 4 addiert, um ihn auf das nächste Element zeigen zu lassen.
\begin{verbatim}
pop:    lw      $r3, 0($sp)
        addi    $sp, $sp, 4
\end{verbatim}
\item \texttt{push}: Hier ist das Gegenteil der Fall. Der Stackpointer wird um -4 verschoben, sodass dort das neue Element eingefügt werden kann.
\begin{verbatim}
push:   addi    $sp, $sp, -4
        sw      $r3, 0($sp)
\end{verbatim}
\end{itemize}

\section{\texttt{loadi}}
Da wir eine 32 Bit Konstante speichern wollen, aber der Befehl insgesamt nur 32 Bit sein kann, muss die Instruktion aufgeteilt werden in zwei separate Befehle.
Mit \texttt{lui} können die 16 oberen Bits gesetzt werden, und mit \texttt{ori} die Unteren.\\
\textit{imm\_upper} sind die höherwertigen 16 Bits, \textit{imm\_lower} analog dazu die Niederwertigen.
\begin{verbatim}
        lui    $r3, imm_upper
        ori    $r3, $r3, imm_lower
\end{verbatim}

\section{ALU: OPCodes}
\begin{center}
\begin{tabular}{|c|c|c|p{8cm}|}\hline
\textbf{operation} & \textbf{\textit{opcode}} & \textbf{\textit{funct}} & \textbf{Erklärung}\\\hline
\texttt{and} & 000 000 & 100 100 & Beim bitweisen and werden \textbf{Ainvert} und \textbf{Binvert} auf 0 gesetzt, und bei der Operation, also beim Multiplexer, wird das erste Resultat weiter verwertet. Vor dem Multiplexer führen die Daten durch ein and-Modul.
\\\hline
\texttt{or} & 000 000 & 100 101 & Die Bits hier sind gleich gesetzt wie beim and, ausser dass beim Multiplexer das zweite Resultat ausgewählt wird (das vorher durch ein OR-Modul geführt wurde).\\\hline
\texttt{add} & 000 000 & 100 000 & Hier sind \textbf{Ainvert} und \textbf{Binvert} auf 0, und beim Multiplexer wird das dritte Resultat verwendet. Dieses Resultat stammt aus einem Halbaddierer.\\\hline
\texttt{sub} & 000 000 & 100 010 & Gleich wie add, nur dass \textbf{Binvert} auf 1 gesetzt wurde.\\\hline
\texttt{slt} & 000 000 & 101 010 & Gleich wie sub, nur dass bei Operation der Wert auf 3 gesetzt wird, so dass das Ergebnis von less als Resultat der ALU weiterverwertet wird.\\\hline
\texttt{nor} & 000 000 & 100 111 & Gleich wie and, nur dass die beiden Schaltungen bei \textbf{Ainvert} und \textbf{Binvert}  auf 1 gesetzt wurden, d.h. es wird mit dem Komplement der beiden weitergerechnet.\\\hline
\end{tabular}
\end{center}


\end{document}
