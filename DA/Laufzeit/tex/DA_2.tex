\documentclass[a4paper,10pt,headlines=3.2]{scrartcl}
\usepackage{graphicx}           %Bilder

%\usepackage[T1]{fontenc}        %Umlaute
%\usepackage[latin1]{inputenc}   %Windows
%\usepackage[utf8x]{inputenc}	%Linux
\usepackage{ucs}

\usepackage[ngerman]{babel}     %Deutsche Sprache
\usepackage{amsmath}            %Math. Zeichen
\usepackage{pifont}             %Skalierbare Schriftart
\usepackage{array}
\usepackage{epsfig}             %Erweiterte Grafiken
\usepackage{makeidx}            %Stichwortverzeichnis
\usepackage[pdftex]{color} 

\newcommand{\changefont}[3]{
\fontfamily{#1} \fontseries{#2} \fontshape{#3} \selectfont}

\makeindex

\usepackage[automark]{scrpage2}
\usepackage[nosectionbib]{apacite}               %Zitieren

%\usepackage[colorlinks]{hyperref}%Hyperlinks

\usepackage{lmodern}
\usepackage{scrpage2}           %KOMA-Script
\usepackage{tipa}
\usepackage{qtree}

\usepackage{remreset}			%Fussnoten global
\makeatletter
\@removefromreset{footnote}{chapter}
\makeatother 

\setcounter{tocdepth}{3}

%Kopfzeilen
\pagestyle{scrheadings}         %Seitenstil scrheadings verwenden

%\setlength{\textheight}{24cm}
%\setlength{\textwidth}{16cm}
%\setlength{\topmargin}{-2cm}
%\setlength{\oddsidemargin}{0cm}

% Groesse des Textbereiches in der Seite
\setlength{\textwidth}{16cm}
\setlength{\textheight}{22cm}
% Kopf- und Fusszeile, Hoehe und Abstand vom Text
\setlength{\headheight}{15pt}
\setlength{\headsep}{0.8cm}
% Linker Seiteneinzug
\setlength{\oddsidemargin}{2.5cm} \addtolength{\oddsidemargin}{-1in}
\setlength{\evensidemargin}{2.5cm} \addtolength{\evensidemargin}{-1in}
% Andere Groessen ausrechnen (vertikal zentrieren)
\setlength{\footskip}{\headsep}
\addtolength{\footskip}{\headheight}
\setlength{\topmargin}{\paperheight}
\addtolength{\topmargin}{-\textheight}
\addtolength{\topmargin}{-\headheight}
\addtolength{\topmargin}{-\headsep}
\addtolength{\topmargin}{-\footskip}
\addtolength{\topmargin}{-2in}
\addtolength{\topmargin}{-0.5\topmargin}

%Schriftart
\changefont{cmss}{m}{n}

%Abstand zur�cksetzen
\setlength{\headheight}{20pt}

\usepackage{listings} 
\lstset{numbers=left, numberstyle=\tiny, numbersep=5pt} \lstset{language=Java} 

\clearscrheadfoot
%\renewcommand{\headheight}{40pt} 
\ihead[]{Datenstrukturen und Algorithmen \\Fr�hlingssemester 2011 \\Institut f�r angewandte Mathematik} % - linke Kopfzeile 
\ohead[asdasd]{�bungsblatt 2 \\Abgabetermin 10. M�rz 2011 \\Adrianus Kleemans [07-111-693]} % - linke Kopfzeile 
\setheadsepline{.4pt} %Separate Linie im Kopf
\cfoot[\pagemark]{\pagemark} %- mittlere Fusszeile 

\begin{document}
\section*{Theoretische Aufgaben}
\subsection*{Aufgabe 1}
Die Definition der $O$-Notation $O(g(n))= \{f(n):\textrm{ es existieren positive Konstanten } c \textrm{ und } n_{0}, \textrm{ sodass }
0 \leq f(n) \leq f(n) \leq c \cdot g(n) \textrm{ f�r alle } n \geq n_{0} \}$
Die $O(n)$-Notation kann auch als obere symptotische Schranke verstanden werden. Sie kann verwendet werden, bis auf einen konstanten Faktor die obere Schranke einer Funktion anzugeben.\\
Bezogen auf die Aufgabe l�sst sich sagen, dass die Aussage deshalb keinen Sinn macht, weil die $O(n)$-Notation eine obere Schranke festlegt, die Formulierung von \textit{mindestens} jedoch dem widerspricht.

\subsection*{Aufgabe 2}
Gem�ss der Definition 
$\Theta(g)= \{f:\textrm{ es existieren positive Konstanten } c_{1}, c_{2} \textrm{ und } n_{0}, \textrm{ sodass } \\
0 \leq c_{1}\cdot g(n) \leq f(n) \leq c_{2} \cdot g(n) \textrm{ f�r alle } n \geq n_{0} \}$ \\
ist die $\Theta(n)$-Notation dadurch definiert, dass es ab einem Wert $n_{0}$ eine obere Schranke $c_{1}\cdot g(n) = O(n)$ gibt, und ebenso eine untere Schranke $\Omega(g(n))$.
Die Laufzeit eines Algorithmus muss deshalb dann genau $\Theta(g(n))$ sein, wenn $O(g(n))$ und $\Omega(g(n))$ auch zutreffen.

\subsection*{Aufgabe 3}
Zeige $a^{log_{b} n} = n^{log_{b} a}$:
\begin{eqnarray}
a^{log_{b} n} = n^{log_{b} a} \textrm{ // } log_{n}\\
log_{n}(a^{log_{b} n}) = log_{n}(n^{log_{b} a})\\
log_{b}(n) \cdot log_{n}(a) = log_{b}(a) \cdot log_{n}(n)\\
log_{b}(n) \cdot \frac{log_{b}(a)}{log_{b}(n)} = log_{b}(a) \\
log_{b}(a) = log_{b}(a)
\end{eqnarray}

\subsection*{Aufgabe 4}
Die Gleichung $\Theta(log_{a} n) = \Theta(log_{b} n)$ stimmt nur, wenn die beiden $\Theta()$-Funktionen um einen konstanten Faktor danebenliegen.\\
Es gilt deshalb zu zeigen: $\Theta(log_{a} n) = c\cdot\Theta(log_{b} n)$.
\begin{eqnarray}
\Theta(log_{a} n) = \Theta(\frac{log_{a} n}{log_{a} b}) \\
\Rightarrow log_{a} b\cdot\Theta(log_{a} n) = \Theta(log_{a} n)
\end{eqnarray}
Da $a$ und $b$ konstant sind, muss auch $log_{a} b$ konstant sein. Damit muss auch $\Theta(log_{a} n) = \Theta(log_{b} n)$ gelten.


\subsection*{Aufgabe 5}
Zeigen Sie, dass die Rekursionsgleichung $T(n) = 2\cdot T([n/4] + 12) + 3n$ die L�sung $O(n\cdot log(n))$ hat.\\\\
Die Induktionsannahme wird in die Gleichung eingesetzt:
\begin{eqnarray}
T(n) = 2\cdot T(\frac{n}{4} + 12) + 3n \leq 2\cdot\frac{n}{4}\cdot log\frac{n}{4} + 3n\\
=\frac{n}{2}\cdot c\cdot log\frac{n}{4} + 3n \\
=\frac{n\cdot c\cdot log_{4} n}{2} - \frac{n\cdot c\cdot log_{4} 4}{2} + 3n\\
\underbrace{c\cdot n\cdot log_{4}(n)}_{Induktionsannahme} \underbrace{- \frac{c\cdot n\cdot log_{4}(n)}{2} - \frac{c\cdot n}{2} + 3n}_{Residuum}
\end{eqnarray}
Ungleichung gilt, falls $c \geq 1$.

\subsection*{Aufgabe 6}
Strukturbaum zu $T(n) = T(n/3) + T(2n/3) + cn$:\\\\
\begin{minipage}{8 cm} 
\Tree [.{$c\cdot n$} 
	    [.{c$\cdot n\cdot\frac{1}{3}$} {$c\cdot n\cdot\frac{1}{3}^2$} {$c\cdot n\cdot\frac{1}{3}\cdot\frac{2}{3}$} ] 
	    [.{c$\cdot n\cdot\frac{2}{3}$} {$c\cdot n\cdot\frac{2}{3}\cdot\frac{1}{3}$} {$c\cdot n\cdot\frac{2}{3}^2$} ] 
      ]

\end{minipage}
\begin{minipage}{8 cm} 
$= c\cdot n$\\\\
$= c\cdot n$\\\\
$= c\cdot n$\\
\end{minipage}
H�he des Baumes: $log(n)$ (da der Algorithmus stoppt, wenn $\frac{1}{3}^{n}=1$).\\
$T(n) = \sum\limits_{i=0}^{h} cn$, wobei die H�he $h = log(n)$\\
$\Rightarrow T(n)= c\cdot n\cdot log(n)$. Dies ist in $\Omega(n\cdot log(n))$.

\subsection*{Aufgabe 7}
\begin{eqnarray}
T(n) = T(n/2) + \Theta(1)\\
a=1,b=2, f(n)=1\\
\textrm{Fall 2: }f(n) = \Theta(n^{log_{b} a})\\
\Rightarrow 1 = \Theta(n^{0})\\
T(n) = \Theta(lg(n)) = \Theta(log(n))
\end{eqnarray}


\subsection*{Aufgabe 8}
\begin{center}
\begin{tabular}{|l|l|l|}
\hline
Zeitkomplexit�t T(n) & Problemgr�sse l�sbar in 10s & Problemgr�sse l�sbar in 1000s\\
\hline
$5n$ & 200 & 20000\\
\hline
$3n^5$ & 3 & 8\\
\hline
$2n^{1.1}$ & 284 & 18697\\
\hline
$5log_{2}(7n)$ & $2.3\cdot 10^{59}$ & $3.9\cdot 10^{6020}$\\
\hline
$2^{4n}$ & 2 & 4 \\
\hline
\end{tabular}
\end{center}

\subsection*{Aufgabe 9}
�ussere Schleife: $log(n)$. Innere Schleife: $n$.\\
Funktion: $\Theta(n\cdot log(n))$. Laufzeit in der Summenformel: $T(n) = \sum\limits_{i=0}^{n} 2^{i-1} = \frac{2^{n+1}-1}{2}$\\
\end{document}
