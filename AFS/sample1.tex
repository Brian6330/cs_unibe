\documentclass[a4paper,10pt,headlines=3.2]{scrartcl}

% Basispakete
\usepackage[utf8]{inputenc}
\usepackage{amsmath}
\usepackage[automark]{scrpage2}
\usepackage{graphicx}
\usepackage{amssymb}
\usepackage{tikz}
\usepackage{ifthen}
\usepackage{calc}
\usepackage{fancybox}
\usepackage{tikz}
\usepackage{ngerman}

\usepackage{graphicx}           %Bilder
\usepackage{amsmath}            %Math. Zeichen
\usepackage{pifont}             %Skalierbare Schriftart
\usepackage{array}
\usepackage{epsfig}             %Erweiterte Grafiken
\usepackage{makeidx}            %Stichwortverzeichnis
\usetikzlibrary{automata,positioning}
\usepackage[left=30mm,right=20mm,top=25mm,bottom=25mm]{geometry}
\usepackage{mdframed}

\newcommand{\changefont}[3]{
\fontfamily{#1} \fontseries{#2} \fontshape{#3} \selectfont}

\makeindex

\usepackage{lmodern}
\usepackage{scrpage2}           %KOMA-Script
\usepackage{tipa}
\usepackage{qtree}

\usepackage{remreset}           %Fussnoten global
\makeatletter
\@removefromreset{footnote}{chapter}
\makeatother 

\setcounter{tocdepth}{3}

%Kopfzeilen
\pagestyle{scrheadings}         %Seitenstil scrheadings verwenden

% Groesse des Textbereiches in der Seite
\setlength{\textwidth}{16cm}
\setlength{\textheight}{21.3cm}
% Kopf- und Fusszeile, Hoehe und Abstand vom Text
\setlength{\headheight}{15pt}
\setlength{\headsep}{0.8cm}
% Linker Seiteneinzug
\setlength{\oddsidemargin}{2.5cm} \addtolength{\oddsidemargin}{-1in}
\setlength{\evensidemargin}{2.5cm} \addtolength{\evensidemargin}{-1in}
% Andere Groessen ausrechnen (vertikal zentrieren)
\setlength{\footskip}{\headsep}
\addtolength{\footskip}{\headheight}
\setlength{\topmargin}{\paperheight}
\addtolength{\topmargin}{-\textheight}
\addtolength{\topmargin}{-\headheight}
\addtolength{\topmargin}{-\headsep}
\addtolength{\topmargin}{-\footskip}
\addtolength{\topmargin}{-2in}
\addtolength{\topmargin}{-0.5\topmargin}

%Abstand zurücksetzen
\setlength{\headheight}{20pt}

\usepackage{listings} 
\lstset{numbers=left, numberstyle=\tiny, numbersep=5pt} \lstset{language=Java} 
\changefont{cmss}{m}{n}

%Tabellen
\usepackage{booktabs}
\usepackage{multirow}

%Kopfzeilen
\clearscrheadfoot
%\renewcommand{\headheight}{40pt} 
\ihead[]{Automaten und formale Sprachen \\Frühlingssemester \\Institut für angewandte
Mathematik} % - linke Kopfzeile 
\ohead[asdasd]{Übungsaufgaben \\ Name \\ Name} % - linke Kopfzeile 
\setheadsepline{.4pt} %Separate Linie im Kopf
\cfoot[\pagemark]{\pagemark} %- mittlere Fusszeile 

\begin{document}
\section{Konstruktion des regulären Ausdrucks}
Die Zustände können wie folgt nummeriert und dargestellt werden:
\begin{center}
\begin{tikzpicture}[scale=1.5, auto,swap]
\foreach \pos/\name/\disp/\initial/\accepting in {  % define nodes: position | name | displayed name | initial? | accepting?
  {(1,0)/q1/q_1/initial/accepting}, 
  {(1,2)/q2/q_2//},
  {(3,1)/q3/q_3//}, 
  {(-1,1)/q4/q_4//}}
\node[state,\initial,\accepting,minimum size=20pt,inner sep=0pt] (\name) at \pos {$\disp$};
\foreach \source/\dest/\name/\pos in { % connect nodes: source | destination | position of text
  q1/q2/a/left,
  q2/q3/b/above,
  q2/q4/c/above,
  q3/q1/c/below,
  q4/q1/b/below}
\path[draw,\pos,thick,->] (\source) -- node {$\name$} (\dest);
\end{tikzpicture}
\end{center}

\begin{align}
A(E) &= r^{4}_{11} + r^{4}_{14} \\
r^{4}_{11} &= r^{3}_{14} (r^{3}_{44})^{*} r^{3}_{41} + r^{3}_{11} \\
r^{3}_{11} &= r^{2}_{13} (r^{2}_{33})^{*} r^{2}_{31} + r^{2}_{11} \\
r^{2}_{11} &= \emptyset \\
&\text{Zwischenresultat: } r^{4}_{11} = r^{3}_{14} (r^{3}_{44})^{*} r^{3}_{41} + r^{2}_{13} (r^{2}_{33})^{*} r^{2}_{31} \\
r^{3}_{14} &= r^{2}_{13} (r^{2}_{33})^{*} r^{2}_{34} + r^{2}_{14} = ab (cab)^{*} cac + ac \\
r^{3}_{44} &= r^{2}_{43} (r^{2}_{33})^{*} r^{2}_{34} + r^{2}_{44} = bab (cab)^{*} cac + bac \\
r^{3}_{41} &= r^{2}_{43} (r^{2}_{33})^{*} r^{2}_{31} + r^{2}_{41} = bab (cab)^{*} c + b \\
r^{2}_{13} &= ab \\
r^{2}_{33} &= cab \\
r^{2}_{31} &= r^{1}_{32} (r^{1}_{22})^{*} r^{1}_{21} + r^{1}_{31} = r^{1}_{31} = c \\
\Rightarrow r^{4}_{11} &= (ab (cab)^{*} cac + ac) (bab (cab)^{*} cac + bac)^{*} (bab (cab)* c + b ) + (ab) (cab)^{*} (c)
\end{align}
Intuitiv würde man meinen, dass
\begin{align}
(abc)^{*} + (acb)^{*}
\end{align}
die Lösung ist, aber gemäss Regeln im Buch kann obiges Resultat (13) hergeleitet werden.

\section{Permutationsgruppe}
a) \\
Mit folgendem Ausdruck kann eine Regelmenge produziert werden, um für beliebige n eine Regelmenge zu erhalten.
\begin{align}
P = \left \{ \bigcup _{i=1}^{n} \bigcup_{j=i+1}^{n} X_i X_j \right \}
\end{align}
Anzahl der Regeln für ein gewisses $n$:
\begin{align}
\sum_{k=1}^{n} 2\cdot (k-1)
\end{align}

Für $n=3$ ergibt dies folgende 6 Produktionen:
\begin{align}
S \rightarrow X_1 X_2 X_3 \\
X_1 X_2 \rightarrow X_2 X_1 \\
X_2 X_1 \rightarrow X_1 X_2 \\
X_1 X_3 \rightarrow X_3 X_1 \\
X_3 X_1 \rightarrow X_1 X_3 \\
X_2 X_3 \rightarrow X_3 X_2 \\
X_3 X_2 \rightarrow X_2 X_3
\end{align}
Beweis der Korrektheit: Mit den entstandenen Produktionsregeln können alle nebeneinander stehenden $X_i X_j$ mit $i, j < n$ beliebig vertauscht werden. Da bei jeder Produktion nicht nur die Anzahl, sondern die genauen Zeichen erhalten bleiben, kann genau die Permutationsgruppe gebildet werden, aber nichts anderes. 
\begin{itemize}
\item Ableitung für 321:
\begin{align}
S \rightarrow X_1 X_2 X_3
X_1 X_2 X_3 \rightarrow X_1 X_3 X_2 \\
X_1 X_3 X_2 \rightarrow X_3 X_1 X_2 \\
X_3 X_1 X_2 \rightarrow X_3 X_2 X_1 \\
X_3 X_2 X_1 \rightarrow 321
\end{align}
\item Ableitung für 312:
\begin{align}
S \rightarrow X_1 X_2 X_3 \\
X_1 X_2 X_3 \rightarrow X_1 X_3 X_2 \\
X_1 X_3 X_2 \rightarrow X_3 X_1 X_2 \\
X_3 X_1 X_2 \rightarrow 312 
\end{align}
\end{itemize}
b) \\
Da die Bedingung gilt, dass der Ausdruck recht die Länge $n-1$ haben soll, ist Typ-3 sicht nicht möglich, da sonst bei jeder Produktion ein terminales Zeichen erzeugt werden müsste, ohne den Rest zu verändern, und dabei noch falsche Produktionen vermieden werden müssten. Bei Typ-2 wäre auch das Problem, dass sie kontextfrei ist, und dabei auch falsche Ausdrücke generiert werden könnten. Evtl. würde in Typ-1 eine Lösung existieren, wenn bestimmte zusätzliche Non-Terminals eingeführt werden würden, die sich nur im Kontext zu anderen Non-Terminals richtig zu Terminals ableiten lassen würden und dabei falsche Produktionen vermeiden. Die Regelmenge dürfte aber beträchtlich steigen. Die obige Grammatik ist Typ-0.

\section{Erzeugung Grammatik}
Analog zu Abbildung 9.1 wird als Ziel $A\hat{B}CBA$ festgelegt (Unterscheidung der beiden $B$s) und folgende Produktionen festgelegt:
\begin{align}
S \rightarrow aS\hat{B} CBA \\
S \rightarrow S\hat{B} \hat{B}C \\
C\hat{B}  \rightarrow \hat{B}C \\
\hat{B}C  \rightarrow  C\hat{B} \\
CB  \rightarrow  BC \\
BA  \rightarrow  AB \\
a\hat{B}  \rightarrow  ab \\
b\hat{B}  \rightarrow  bb \\
bC  \rightarrow  bc \\
cC   \rightarrow cc \\
cB  \rightarrow  cb \\
bB  \rightarrow bb \\
bA  \rightarrow ba \\
aA  \rightarrow aa
\end{align}
Damit lässt sich zu Beginn (wie in der Abbildung) soweit wie nötig expandieren (über $S \rightarrow aS\hat{B} CBA$), bis die gewünschte Länge erreicht ist, und danach mit den Produktionen bis zu den Terminals ($a^n b^n c^n b^n a^n$) auflösen.

\end{document}