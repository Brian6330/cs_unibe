\documentclass[a4paper,10pt,headlines=3.2]{scrartcl}

% Basispakete
\usepackage[utf8]{inputenc}
\usepackage{amsmath}
\usepackage[automark]{scrpage2}
\usepackage{graphicx}
\usepackage{amssymb}
\usepackage{tikz}
\usepackage{ifthen}
\usepackage{calc}
\usepackage{fancybox}
\usepackage{tikz}
\usepackage{ngerman}

\usepackage{graphicx}           %Bilder
\usepackage{amsmath}            %Math. Zeichen
\usepackage{pifont}             %Skalierbare Schriftart
\usepackage{array}
\usepackage{epsfig}             %Erweiterte Grafiken
\usepackage{makeidx}            %Stichwortverzeichnis
\usetikzlibrary{automata,positioning}
\usepackage[left=30mm,right=20mm,top=25mm,bottom=25mm]{geometry}
\usepackage{mdframed}

\newcommand{\changefont}[3]{
\fontfamily{#1} \fontseries{#2} \fontshape{#3} \selectfont}

\makeindex

\usepackage[automark]{scrpage2}
\usepackage[nosectionbib]{apacite}               %Zitieren

\usepackage{lmodern}
\usepackage{scrpage2}           %KOMA-Script
\usepackage{tipa}
\usepackage{qtree}

\usepackage{remreset}			%Fussnoten global
\makeatletter
\@removefromreset{footnote}{chapter}
\makeatother 

\setcounter{tocdepth}{3}

%Kopfzeilen
\pagestyle{scrheadings}         %Seitenstil scrheadings verwenden

% Groesse des Textbereiches in der Seite
\setlength{\textwidth}{16cm}
\setlength{\textheight}{21.3cm}
% Kopf- und Fusszeile, Hoehe und Abstand vom Text
\setlength{\headheight}{15pt}
\setlength{\headsep}{0.8cm}
% Linker Seiteneinzug
\setlength{\oddsidemargin}{2.5cm} \addtolength{\oddsidemargin}{-1in}
\setlength{\evensidemargin}{2.5cm} \addtolength{\evensidemargin}{-1in}
% Andere Groessen ausrechnen (vertikal zentrieren)
\setlength{\footskip}{\headsep}
\addtolength{\footskip}{\headheight}
\setlength{\topmargin}{\paperheight}
\addtolength{\topmargin}{-\textheight}
\addtolength{\topmargin}{-\headheight}
\addtolength{\topmargin}{-\headsep}
\addtolength{\topmargin}{-\footskip}
\addtolength{\topmargin}{-2in}
\addtolength{\topmargin}{-0.5\topmargin}

%Abstand zurücksetzen
\setlength{\headheight}{20pt}

\usepackage{listings} 
\lstset{numbers=left, numberstyle=\tiny, numbersep=5pt} \lstset{language=Java} 
\changefont{cmss}{m}{n}

%Tabellen
\usepackage{booktabs}
\usepackage{multirow}

%Kopfzeilen
\clearscrheadfoot
\ihead[]{Automaten und formale Sprachen \\Frühlingssemester \\Institut für angewandte
Mathematik} % - linke Kopfzeile 
\ohead[asdasd]{Übungsaufgaben \\ Name \\ Name} % - linke Kopfzeile 
\setheadsepline{.4pt} %Separate Linie im Kopf
\cfoot[\pagemark]{\pagemark} %- mittlere Fusszeile 

\begin{document}
\section{Aufgabe}
\begin{align}
G = (T, N, S, P) \\
T = \{a, b, c\} \\
N = \{ S, X\} \\
P = \{ S &\mapsto X \\
X &\mapsto XX \\
X &\mapsto aa \\
X &\mapsto bb \\
X &\mapsto cc \}
\end{align}
Konstruktion eines DEA:
\begin{align}
\Sigma &= T \\
Q &= N \cup \{q_2\} = \{q_0, q_1, q_2, q_3\} \\
q_0 &= S \\
F &= \{q_f\} = q_0 \\
\delta(q_0, a) &= q_1 \\
\delta(q_1, a) &= q_1 \\
\delta(q_0, b) &= q_1 \\
\delta(q_2, b) &= q_1 \\
\delta(q_0, c) &= q_1 \\
\delta(q_3, c) &= q_1 \\
\end{align}

\begin{center}
\begin{tikzpicture}[scale=1.5, auto,swap]
\foreach \pos/\name/\disp/\initial/\accepting in {  % define nodes: position | name | displayed name | initial? | accepting?
  {(0,1)/q0/q_0/initial/accepting}, 
  {(2,0)/q3/q_3//},
  {(3,1)/q2/q_2//}, 
  {(2,2)/q1/q_1//}}
\node[state,\initial,\accepting,minimum size=20pt,inner sep=0pt] (\name) at \pos {$\disp$};
\foreach \source/\dest/\name/\pos in { % connect nodes: source | destination | position of text
  q0/q1/a/above,
  q1/q0/a/below,
  q0/q2/b/above,
  q2/q0/b/below,
  q0/q3/c/above,
  q3/q0/c/below}
\path[draw,\pos,thick,->] (\source) -- node {$\name$} (\dest);
\end{tikzpicture}
\end{center}

\section{Aufgabe}
\begin{center}
\begin{tabular}{c|c|c|c|}
\textbf{} & \textbf{a} & \textbf{b} & \textbf{c}\\\hline
\textbf{$q_0$} & $q_1$ & $q_3$ & $q_5$\\\hline
\textbf{$q_1$} & - & $q_2$ & $q_6$\\\hline
\textbf{$q_2$} & - & - & $q_0$\\\hline
\textbf{$q_3$} & $q_2$ & - & $q_4$\\\hline
\textbf{$q_4$} & $q_0$ & - & -\\\hline
\textbf{$q_5$} & $q_6$ & $q_4$ & -\\\hline
\textbf{$q_6$} & - & $q_0$ & -\\\hline
\end{tabular}
\end{center}

\begin{align}
G &= (T, N, S, P) \\
T &= \{a, b, c\} \\
N &= \{S, A, B, C, D, E, F\} \\
P = \{ S &\mapsto aA \\
A &\mapsto bD \\
A &\mapsto cE \\
D &\mapsto cS \\
D &\mapsto c \\
S &\mapsto bB \\
B &\mapsto aD \\
B &\mapsto cF \\
E &\mapsto bS \\
E &\mapsto b \\
S &\mapsto cC \\
C &\mapsto aE \\
C &\mapsto bF \\
F &\mapsto aS \\
F &\mapsto a \}
\end{align}

\end{document}