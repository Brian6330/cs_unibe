\documentclass[a4paper,10pt,headlines=3.2]{scrartcl}
\usepackage{graphicx}           %Bilder

%\usepackage[T1]{fontenc}        %Umlaute
%\usepackage[latin1]{inputenc}   %Windows
%\usepackage[utf8x]{inputenc}	%Linux
\usepackage{ucs}

\usepackage[ngerman]{babel}     %Deutsche Sprache
\usepackage{amsmath}            %Math. Zeichen
\usepackage{pifont}             %Skalierbare Schriftart
\usepackage{array}
\usepackage{epsfig}             %Erweiterte Grafiken
\usepackage{makeidx}            %Stichwortverzeichnis
\usepackage[pdftex]{color} 

\newcommand{\changefont}[3]{
\fontfamily{#1} \fontseries{#2} \fontshape{#3} \selectfont}

\makeindex

\usepackage[automark]{scrpage2}
\usepackage[nosectionbib]{apacite}               %Zitieren

%\usepackage[colorlinks]{hyperref}%Hyperlinks

\usepackage{lmodern}
\usepackage{scrpage2}           %KOMA-Script
\usepackage{tipa}
\usepackage{qtree}
\usepackage{wasysym}

\usepackage{remreset}			%Fussnoten global
\makeatletter
\@removefromreset{footnote}{chapter}
\makeatother 

\setcounter{tocdepth}{3}

%Kopfzeilen
\pagestyle{scrheadings}         %Seitenstil scrheadings verwenden

%\setlength{\textheight}{24cm}
%\setlength{\textwidth}{16cm}
%\setlength{\topmargin}{-2cm}
%\setlength{\oddsidemargin}{0cm}

% Groesse des Textbereiches in der Seite
\setlength{\textwidth}{16cm}
\setlength{\textheight}{22cm}
% Kopf- und Fusszeile, Hoehe und Abstand vom Text
\setlength{\headheight}{15pt}
\setlength{\headsep}{0.8cm}
% Linker Seiteneinzug
\setlength{\oddsidemargin}{2.5cm} \addtolength{\oddsidemargin}{-1in}
\setlength{\evensidemargin}{2.5cm} \addtolength{\evensidemargin}{-1in}
% Andere Groessen ausrechnen (vertikal zentrieren)
\setlength{\footskip}{\headsep}
\addtolength{\footskip}{\headheight}
\setlength{\topmargin}{\paperheight}
\addtolength{\topmargin}{-\textheight}
\addtolength{\topmargin}{-\headheight}
\addtolength{\topmargin}{-\headsep}
\addtolength{\topmargin}{-\footskip}
\addtolength{\topmargin}{-2in}
\addtolength{\topmargin}{-0.5\topmargin}

%Abstand zur�cksetzen
\setlength{\headheight}{20pt}

\usepackage{listings} 
\lstset{numbers=left, numberstyle=\tiny, numbersep=5pt} \lstset{language=Java} 
\changefont{cmss}{m}{n}

\clearscrheadfoot
%\renewcommand{\headheight}{40pt} 
\ihead[]{Datenbanken \\Fr�hlingssemester 2011 \\Institut f�r angewandte
Mathematik} % - linke Kopfzeile 
\ohead[asdasd]{�bung 7, Abgabe 12. April 2011 \\Adrianus Kleemans
[07-111-693]\\Pinar Kayalar [10-123-453]} % - linke Kopfzeile 
\setheadsepline{.4pt} %Separate Linie im Kopf
\cfoot[\pagemark]{\pagemark} %- mittlere Fusszeile

\begin{document}
\section*{Aufgabe 1}
\begin{itemize}
 \item Geben Sie eine Liste aller Autoren aus.
\begin{verbatim}
SELECT au_lname, au_fname, phone, address, city, state, country, postalcode 
FROM authors
\end{verbatim}
 \item Geben Sie eine Liste aller Titel aus.
\begin{verbatim}
SELECT title, type, price, advance, total_sales, notes, pubdate, contract
FROM titles
\end{verbatim}
 \item Gibt es Autoren, die in San Francisco wohnen? Welche?
\begin{verbatim}
SELECT au_lname, au_fname, phone, address, city, state, country, postalcode 
FROM authors
WHERE city = "San Francisco"
\end{verbatim}
 \item Wieviele Titel sind gespeichert?
\begin{verbatim}
SELECT count(title) as AnzahlTitel
FROM titles
\end{verbatim}
 \item Wieviele Autoren sind gespeichert?
\begin{verbatim}
SELECT count(au_id) as AnzahlAutoren
FROM authors
\end{verbatim}
 \item Geben Sie eine Liste aller Titel die mit 'S' beginnen aus.
\begin{verbatim}
SELECT title, type, price, advance, total_sales, notes, pubdate, contract
FROM titles
WHERE title LIKE 'S%'
\end{verbatim}
 \item Bestimmen Sie den durchschnittlichen Preis eines Titels.
\begin{verbatim}
SELECT avg(price) as DurchschnPreis
FROM titles
\end{verbatim}
 \item Geben Sie das Datum aller Verk�ufe des Ladens 'Bookbeat' an.
\begin{verbatim}
SELECT date
FROM stores, sales
WHERE sales.stor_id = stores.stor_id AND stor_name = 'Bookbeat'
\end{verbatim}
 \item Geben Sie alle Titel aus, die im Laden 'Bookbeat' verkauft wurden.
\begin{verbatim}
SELECT title, type, price, advance, total_sales, notes, pubdate, contract
FROM stores, sales, salesdetail, titles
WHERE sales.stor_id = stores.stor_id AND sales.ord_num = salesdetail.ord_num 
      AND salesdetail.title_id = titles.title_id AND stor_name = 'Bookbeat'
\end{verbatim}
 \item Geben Sie den Discounttype jedes Ladens, inklusive Namen und Ort des Ladens an.
\begin{verbatim}
SELECT discounttype, stor_name, stor_adress, city, state, country, postalcode
FROM stores, discounts
WHERE stores.stor_id = discounts.stor_id
\end{verbatim}
\end{itemize}

\section*{Aufgabe 2}
\begin{itemize}
 \item Die angegebene Anfrage hat auf einer MySQL-Umgebung funktioniert und g�ltige Resultate zur�ckgegeben. Die
Aufgabe k�nnte auf einem Namenskonflikt beruhen.
 \item Korrigierter Ausdruck:
\begin{verbatim}
SELECT title, price
FROM titles
WHERE price >= ALL (	SELECT price AS p 
			FROM titles)
\end{verbatim}
 \item Alle B�cher, welche teuerer als das billigste Psychologie-Buch sind:
\begin{verbatim}
SELECT title, price
FROM titles
WHERE price >= SOME (	SELECT price AS p
			FROM titles
			WHERE type = 'Psychology')
\end{verbatim}
 \item Autoren in einem Staat ohne Store:
\begin{verbatim}
SELECT title, price
FROM authors
WHERE NOT EXISTS (	SELECT city
			FROM stores
			WHERE stores.city = authors.city)
\end{verbatim}
\end{itemize}

\section*{Aufgabe 3}
\begin{itemize}
 \item B�cher, nach Preis sortiert:
\begin{verbatim}
SELECT title, type, price, advance, total_sales, notes, pubdate, contract
FROM titles
ORDER BY price DESC
\end{verbatim}
 \item Autoren, nach Statt, Stadt und Name sortiert.
\begin{verbatim}
SELECT au_lname, au_fname, phone, address, city, state, country, postalcode 
FROM authors
ORDER BY state DESC, city ASC, au_lname ASC 
\end{verbatim}

 \item B�cherarten, zusammen mit Anzahl B�chern:
\begin{verbatim}
SELECT type, count(title)
FROM titles
GROUP BY type
\end{verbatim}
 \item Anzahl Autoren pro Staat:
\begin{verbatim}
SELECT state, count(au_id)
FROM authors
GROUP BY state
ORDER BY count(au_id)
\end{verbatim}

 \item B�cherarten mit mehr als 2 verschiedene B�cher:
\begin{verbatim}
SELECT type, count(title)
FROM titles
GROUP BY type
HAVING count(title) > 2
\end{verbatim}
 \item Alle Publisher, welche weniger B�cher herausgegeben haben als der Durchschnitt.
\begin{verbatim}
SELECT pub_name, count(titles.pub_id)
FROM publishers, titles
WHERE titles.pub_id = publishers.pub_id
GROUP BY pub_name
HAVING count(titles.pub_id) < avg(count(titles.pub_id))
\end{verbatim}
\end{itemize}


\end{document}