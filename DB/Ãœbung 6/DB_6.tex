\documentclass[a4paper,10pt,headlines=3.2]{scrartcl}
\usepackage{graphicx}           %Bilder

%\usepackage[T1]{fontenc}        %Umlaute
%\usepackage[latin1]{inputenc}   %Windows
%\usepackage[utf8x]{inputenc}	%Linux
\usepackage{ucs}

\usepackage[ngerman]{babel}     %Deutsche Sprache
\usepackage{amsmath}            %Math. Zeichen
\usepackage{pifont}             %Skalierbare Schriftart
\usepackage{array}
\usepackage{epsfig}             %Erweiterte Grafiken
\usepackage{makeidx}            %Stichwortverzeichnis
\usepackage[pdftex]{color} 

\newcommand{\changefont}[3]{
\fontfamily{#1} \fontseries{#2} \fontshape{#3} \selectfont}

\makeindex

\usepackage[automark]{scrpage2}
\usepackage[nosectionbib]{apacite}               %Zitieren

%\usepackage[colorlinks]{hyperref}%Hyperlinks

\usepackage{lmodern}
\usepackage{scrpage2}           %KOMA-Script
\usepackage{tipa}
\usepackage{qtree}
\usepackage{wasysym}

\usepackage{remreset}			%Fussnoten global
\makeatletter
\@removefromreset{footnote}{chapter}
\makeatother 

\setcounter{tocdepth}{3}

%Kopfzeilen
\pagestyle{scrheadings}         %Seitenstil scrheadings verwenden

%\setlength{\textheight}{24cm}
%\setlength{\textwidth}{16cm}
%\setlength{\topmargin}{-2cm}
%\setlength{\oddsidemargin}{0cm}

% Groesse des Textbereiches in der Seite
\setlength{\textwidth}{16cm}
\setlength{\textheight}{22cm}
% Kopf- und Fusszeile, Hoehe und Abstand vom Text
\setlength{\headheight}{15pt}
\setlength{\headsep}{0.8cm}
% Linker Seiteneinzug
\setlength{\oddsidemargin}{2.5cm} \addtolength{\oddsidemargin}{-1in}
\setlength{\evensidemargin}{2.5cm} \addtolength{\evensidemargin}{-1in}
% Andere Groessen ausrechnen (vertikal zentrieren)
\setlength{\footskip}{\headsep}
\addtolength{\footskip}{\headheight}
\setlength{\topmargin}{\paperheight}
\addtolength{\topmargin}{-\textheight}
\addtolength{\topmargin}{-\headheight}
\addtolength{\topmargin}{-\headsep}
\addtolength{\topmargin}{-\footskip}
\addtolength{\topmargin}{-2in}
\addtolength{\topmargin}{-0.5\topmargin}

%Abstand zur�cksetzen
\setlength{\headheight}{20pt}

\usepackage{listings} 
\lstset{numbers=left, numberstyle=\tiny, numbersep=5pt} \lstset{language=Java} 
\changefont{cmss}{m}{n}

\clearscrheadfoot
%\renewcommand{\headheight}{40pt} 
\ihead[]{Datenbanken \\Fr�hlingssemester 2011 \\Institut f�r angewandte
Mathematik} % - linke Kopfzeile 
\ohead[asdasd]{�bung 6, Abgabe 5. April 2011 \\Adrianus Kleemans
[07-111-693]\\Pinar Kayalar [10-123-453]} % - linke Kopfzeile 
\setheadsepline{.4pt} %Separate Linie im Kopf
\cfoot[\pagemark]{\pagemark} %- mittlere Fusszeile 


\begin{document}
\section*{Aufgabe 1}
Abfragen:
\begin{itemize}
 \item W�hle alle Titel mit dem Preis aus, welche aus der Sparte Psychologie kommen und zwischen 7 und 19.99 \$
kosten\footnote{Da die Abfrage 'umgangssprachlich' gestaltet werden soll, wird hier bereits eine W�hrung impliziert,
obwohl eigentlich nur der Wert des Attributs \texttt{price}, dessen Dom�ne unbekannt ist, zwischen 7 und 19.99 sein
muss.}.
 \item W�hle alle Titel mit Preis aus, welche eine aktuelle Verkaufszahl zwischen der H�chstmenge und Niedrigstmenge
einer bestimmten Auflage haben.
\end{itemize}
Korrekturen:
\begin{itemize}
 \item Die Attribute \texttt{city} und \texttt{state} kommen mehrfach vor, weshalb es Kollisionen mit den Namen gibt. Zudem sind ein einige Attribute falsch
geschrieben (\texttt{au-lname} statt \texttt{au\_lname}, weiter \texttt{au-fname}, \texttt{pub-name}). Auch die
Tabelle \texttt{titleauthors} gibt es nicht (korrigiert zu $\rightarrow$\texttt{titleauthor}).\\
$\pi_{au\_lname, au\_fname, pub\_name}((((\pi_{au\_lname, au\_fname, au\_id}(authors))\bowtie titleauthor)\bowtie
titles)\bowtie publishers)$

 \item Die Attribute $pub_id, notitles, pub-id, title-id$ sind falsch. Korrigiert man diese, sollte die Anfrage klappen.
\end{itemize}

\section*{Aufgabe 2}
\begin{itemize}
 \item Angenommen Pr�dikat bezieht sich auf  s: $t \in s, t$ erf�llt $p$
\begin{eqnarray}
\sigma_{p}(r \times s) \rightarrow t \in s \\
\sigma_{p}(r) \rightarrow t \notin r \\
\sigma_{p}(r) \times s \\
\textrm{(kein Element von r)} \times s \neq \sigma_{p}(r \times s)
\end{eqnarray}

 \item $x \in B$, r(A,B) - $y \in A$, s(B). \\
F�r jedes $x \in B$ gilt $x \in B(r)$. \\
$r\div s \rightarrow y$ Elemente die alle $x$ von $B(s)$ enthalten.\\
$r\div s\times r_s \rightarrow \vee y \in A$ und $\vee x \in B$.\\
$\rightarrow$ alle $y$ von A die nicht alle $x$ von B enthalten fehlen.\\
$r \div s \times s = r$ wenn alle r nur y enth�lt mit allen $x$ Elementen von $B(s)$.
Wenn nicht alle $x\in B(s)$ auch Element von $B(r)$ sind dann kein $y$  alle $x$ enthalten $\rightarrow$ keine L�sung.

\end{itemize}

\section*{Aufgabe 3}
\begin{itemize}
 \item $4'000\cdot4'000 = 16'000'000 \rightarrow \cdot \frac{1}{4} = 4'000'000 \rightarrow \cdot \frac{1}{4} =
1'000'000$ 
 \item $\frac{1}{4}\cdot 4'000 = 1'000 \rightarrow \cdot 4'000 = 4'000'000 \rightarrow \cdot\frac{1}{4} = 1'000'000$ 
\end{itemize}
Die zweite Abfrage ist effizienter, da viel kleinere tempor�re Tabellen erstellt werden. Es braucht nicht Platz f�r
16'000'000, sondern nur f�r 4'000'000 Eintr�ge. 

\section*{Aufgabe 4}
Eine einfache Abfrage in folgender Form gen�gt: $_{DreieckId}G_{avg(x),avg(x),avg(z)}$.

\end{document}