\documentclass[a4paper,10pt,headlines=3.2]{scrartcl}
\usepackage{graphicx}           %Bilder

%\usepackage[T1]{fontenc}        %Umlaute
%\usepackage[latin1]{inputenc}   %Windows
%\usepackage[utf8x]{inputenc}	%Linux
\usepackage{ucs}

\usepackage[ngerman]{babel}     %Deutsche Sprache
\usepackage{amsmath}            %Math. Zeichen
\usepackage{pifont}             %Skalierbare Schriftart
\usepackage{array}
\usepackage{epsfig}             %Erweiterte Grafiken
\usepackage{makeidx}            %Stichwortverzeichnis
\usepackage[pdftex]{color} 

\newcommand{\changefont}[3]{
\fontfamily{#1} \fontseries{#2} \fontshape{#3} \selectfont}

\makeindex

\usepackage[automark]{scrpage2}
\usepackage[nosectionbib]{apacite}               %Zitieren

%\usepackage[colorlinks]{hyperref}%Hyperlinks

\usepackage{lmodern}
\usepackage{scrpage2}           %KOMA-Script
\usepackage{tipa}
\usepackage{qtree}
\usepackage{wasysym}

\usepackage{remreset}			%Fussnoten global
\makeatletter
\@removefromreset{footnote}{chapter}
\makeatother 

\setcounter{tocdepth}{3}

%Kopfzeilen
\pagestyle{scrheadings}         %Seitenstil scrheadings verwenden

%\setlength{\textheight}{24cm}
%\setlength{\textwidth}{16cm}
%\setlength{\topmargin}{-2cm}
%\setlength{\oddsidemargin}{0cm}

% Groesse des Textbereiches in der Seite
\setlength{\textwidth}{16cm}
\setlength{\textheight}{22cm}
% Kopf- und Fusszeile, Hoehe und Abstand vom Text
\setlength{\headheight}{15pt}
\setlength{\headsep}{0.8cm}
% Linker Seiteneinzug
\setlength{\oddsidemargin}{2.5cm} \addtolength{\oddsidemargin}{-1in}
\setlength{\evensidemargin}{2.5cm} \addtolength{\evensidemargin}{-1in}
% Andere Groessen ausrechnen (vertikal zentrieren)
\setlength{\footskip}{\headsep}
\addtolength{\footskip}{\headheight}
\setlength{\topmargin}{\paperheight}
\addtolength{\topmargin}{-\textheight}
\addtolength{\topmargin}{-\headheight}
\addtolength{\topmargin}{-\headsep}
\addtolength{\topmargin}{-\footskip}
\addtolength{\topmargin}{-2in}
\addtolength{\topmargin}{-0.5\topmargin}

%Abstand zur�cksetzen
\setlength{\headheight}{20pt}

\usepackage{listings} 
\lstset{numbers=left, numberstyle=\tiny, numbersep=5pt} \lstset{language=Java} 
\changefont{cmss}{m}{n}

\clearscrheadfoot
%\renewcommand{\headheight}{40pt} 
\ihead[]{Datenbanken \\Fr�hlingssemester 2011 \\Institut f�r angewandte
Mathematik} % - linke Kopfzeile 
\ohead[asdasd]{�bung 8, Abgabe 19. April 2011 \\Adrianus Kleemans
[07-111-693]\\Pinar Kayalar [10-123-453]} % - linke Kopfzeile 
\setheadsepline{.4pt} %Separate Linie im Kopf
\cfoot[\pagemark]{\pagemark} %- mittlere Fusszeile

\begin{document}
\section*{Aufgabe 1}
\begin{itemize}
 \item Bestimmen Sie alle B�cher, die den gleichen Typ besitzen wie das Buch 'Net Etiquette'.
\begin{verbatim}
SELECT title, type, price, advance, total_sales, notes, pubdate, contract
FROM titles
WHERE type = (SELECT type
                FROM titles AS s
                WHERE s.title = 'Net Etiquette')
\end{verbatim}

 \item Bestimmen Sie f�r jeden Laden den Gesamtpreis der verkauften B�cher jedes Autors
(ohne Discounts), absteigend sortiert.
\begin{verbatim}
SELECT stor_name, sum(salesdetail.qty * titles.price) as sum_sale
FROM titles, sales, salesdetail, stores
GROUP BY stor_name
HAVING stores.stor_id = sales.stor_id 
        AND sales.title_id = salesdetail.title_id
        AND salesdetail.title_id = titles.title_id
ORDER BY sum_sale DESC
\end{verbatim}

 \item Bestimmen Sie alle Autoren, die weniger 'psychology' B�cher herausgegeben haben als 'Livia Karsen'.
\begin{verbatim}
SELECT au_lname, au_fname
FROM authors, title
WHERE authors.au_id = titleauthor.au_id 
      AND titleauthor.title_id = titles.title_id 
      AND total_sales < (SELECT sum(total_sales)
                         FROM titles, titleauthor, author
                         WHERE authors.au_id = titleauthor.au_id 
                         AND titleauthor.title_id = titles.title_id 
                         AND au_lname = 'Karsen' AND au_fname = 'Livia')
\end{verbatim}

 \item Geben Sie f�r jedes Buch seinen Titel, Preis, Anzahl bisher verkaufter Exemplare sowie die effektiv geschuldeten
Tantiemen an. (Die pro Buch geschuldete Tantieme in Prozent des Verkaufspreises ist in der Spalte royalty in
Abh�ngigkeit der Verkaufszahlen angegeben.)
\begin{verbatim}
SELECT title, sum(price * total_sales * royalty)
FROM titles, roysched
WHERE titles.title_id = roysched.title_id
GROUP BY title
\end{verbatim}

 \item Geben Sie die St�dte an, in denen es sowohl Autoren wie auch Verleger gibt.
\begin{verbatim}
SELECT a.city
FROM authors as a, publishers as p
WHERE a.city = p.city
\end{verbatim}

 \item ''Testen Sie, ob die Werte des Attributs total\_sales korrekt sind, indem Sie alle Buchtitel mit falschen Werte
zusammen mit den Werten ausgeben.''\\
Diese Aussage (''Buchtitel mit falsche Werte zusammen mit den Werten'') ergibt keinen Sinn.

 \item Geben Sie die L�den an, die schon B�cher aller Typen verkauft haben.
\begin{verbatim}
SELECT stor_name
FROM stores, salesdetail, titles
GROUP BY stor_name
HAVING stores.stor_id = sales.stor_id
        AND sales.stor_id = salesdetail.stor_id
        AND salesdetail.stor_id = titles.stor_id
        AND count(type) = (SELECT DISTINCT count(type)
                           FROM titles AS t)
\end{verbatim}

\end{itemize}

\section*{Aufgabe 2}
\begin{itemize}
 \item Nein, der doppelte Pfeil zeigt an, dass es zu jedem Server ein Administrator geben muss.
 \item Nein, der doppelte Pfeil zeigt an, dass jeder Administrator genau 1 Server verwaltet.
 \item Nicht notwendigerweise, dies ist auf jeden fall nicht zwingend vorgesehen (nur einfache Linien zwischen dem
Nutzer und den Dokumenten).
 \item Zwei Angestellte, ein Redakteur und ein Designer.
 \item Nicht in der Rolle des Redakteur, f�r diesen ist keine Nutzung vorgesehen. Wenn ein Redakteur als 'Person' jedoch
auch ein Nutzer ist, kann er die Dokumente auch nutzen.
 \item Nur wenn er als Angestellter gleichzeitig auch Redakteur ist (also auch Teil der Redakteurentit�t).
 \item Ein Server kann, muss aber nicht mehrere Dokumente enthalten (aber mindestens eines).
 \item Die Nutzer-Server Beziehung. Jeder Nutzer, der auf Dokumente des Servers zugreift, greift damit auch auf einen
bestimmten Server zu. Es ist aber anzunehmen, dass die Nutzer keinen anderen Zugang zu den Server haben als auf die
Dokumente zuzugreifen.
\end{itemize}


\section*{Aufgabe 3}


\end{document}