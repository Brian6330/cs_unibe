\documentclass[a4paper,10pt,headlines=3.2]{scrartcl}
\usepackage{graphicx}           %Bilder

%\usepackage[T1]{fontenc}        %Umlaute
%\usepackage[latin1]{inputenc}   %Windows
%\usepackage[utf8x]{inputenc}	%Linux
\usepackage{ucs}

\usepackage[ngerman]{babel}     %Deutsche Sprache
\usepackage{amsmath}            %Math. Zeichen
\usepackage{pifont}             %Skalierbare Schriftart
\usepackage{array}
\usepackage{epsfig}             %Erweiterte Grafiken
\usepackage{makeidx}            %Stichwortverzeichnis
\usepackage[pdftex]{color} 

\newcommand{\changefont}[3]{
\fontfamily{#1} \fontseries{#2} \fontshape{#3} \selectfont}

\makeindex

\usepackage[automark]{scrpage2}
\usepackage[nosectionbib]{apacite}               %Zitieren

%\usepackage[colorlinks]{hyperref}%Hyperlinks

\usepackage{lmodern}
\usepackage{scrpage2}           %KOMA-Script
\usepackage{tipa}
\usepackage{qtree}
\usepackage{wasysym}

\usepackage{remreset}			%Fussnoten global
\makeatletter
\@removefromreset{footnote}{chapter}
\makeatother 

\setcounter{tocdepth}{3}

%Kopfzeilen
\pagestyle{scrheadings}         %Seitenstil scrheadings verwenden

%\setlength{\textheight}{24cm}
%\setlength{\textwidth}{16cm}
%\setlength{\topmargin}{-2cm}
%\setlength{\oddsidemargin}{0cm}

% Groesse des Textbereiches in der Seite
\setlength{\textwidth}{16cm}
\setlength{\textheight}{22cm}
% Kopf- und Fusszeile, Hoehe und Abstand vom Text
\setlength{\headheight}{15pt}
\setlength{\headsep}{0.8cm}
% Linker Seiteneinzug
\setlength{\oddsidemargin}{2.5cm} \addtolength{\oddsidemargin}{-1in}
\setlength{\evensidemargin}{2.5cm} \addtolength{\evensidemargin}{-1in}
% Andere Groessen ausrechnen (vertikal zentrieren)
\setlength{\footskip}{\headsep}
\addtolength{\footskip}{\headheight}
\setlength{\topmargin}{\paperheight}
\addtolength{\topmargin}{-\textheight}
\addtolength{\topmargin}{-\headheight}
\addtolength{\topmargin}{-\headsep}
\addtolength{\topmargin}{-\footskip}
\addtolength{\topmargin}{-2in}
\addtolength{\topmargin}{-0.5\topmargin}

%Abstand zur�cksetzen
\setlength{\headheight}{20pt}

\usepackage{listings} 
\lstset{numbers=left, numberstyle=\tiny, numbersep=5pt} \lstset{language=Java} 
\changefont{cmss}{m}{n}

\clearscrheadfoot
%\renewcommand{\headheight}{40pt} 
\ihead[]{Datenbanken \\Fr�hlingssemester 2011 \\Institut f�r angewandte
Mathematik} % - linke Kopfzeile 
\ohead[asdasd]{�bung 5, Abgabe 29. M�rz 2011 \\Adrianus Kleemans
[07-111-693]\\Pinar Kayalar [10-123-453]} % - linke Kopfzeile 
\setheadsepline{.4pt} %Separate Linie im Kopf
\cfoot[\pagemark]{\pagemark} %- mittlere Fusszeile 


\begin{document}
\section*{Aufgabe 1}
Evaluation der Relation $r$:\\
\begin{center}
\begin{tabular}{|c|c|c|c|l}\cline{1-4}
\textbf{A} & \textbf{B} & \textbf{C} & \textbf{D} & \textbf{Evaluation}\\\cline{1-4}
''A'' & 1000 & 3 & '''' & true\\\cline{1-4}
''A'' & 700 & NULL & ''agh'' & unknown\\\cline{1-4}
''A'' & NULL & 0 & ''abcdf'' & unknown\\\cline{1-4}
''A'' & 1000 & 4 & NULL & true\\\cline{1-4}
''B'' & NULL & NULL & ''bdf'' & unknown\\\cline{1-4}
''B'' & 1500 & NULL & ''c'' & unknown\\\cline{1-4}
NULL & 1000 & 8 & '''' & false\\\cline{1-4}
NULL & 700 & 12 & NULL & true\\\cline{1-4}
\end{tabular}
\end{center}



\begin{itemize}
 \item $\sigma_{B\cdot C < 5000 \textrm{ or $D$ is unknown}}(r)$
\begin{center}
\begin{tabular}{|c|c|c|c|}\hline
\textbf{A} & \textbf{B} & \textbf{C} & \textbf{D}\\\hline
''A'' & 1000 & 3 & ''''\\\hline
''A'' & 1000 & 4 & Null\\\hline
Null & 700 & 12 & Null\\\hline
\end{tabular}
\end{center}
 \item $_{A}g_{avg(B),sum(C)}(r)$
\begin{center}
\begin{tabular}{|c|c|c|}\hline
\textbf{A} & \textbf{avg(B)} & \textbf{sum(C)}\\\hline
''A'' & 675 & 7\\\hline
''B'' & 750 & 0\\\hline
\end{tabular}
\end{center}

 \item $_{A}g_{avg(B)}(\pi_{A,B}(r))$
\begin{center}
\begin{tabular}{|c|c|}\hline
\textbf{A} & \textbf{avg(B)}\\\hline
''A'' & 566.66\\\hline
''B'' & 750\\\hline
\end{tabular}
\end{center}

 \item natural join
\begin{center}
\begin{tabular}{|c|c|c|c|c|}\hline
\textbf{A} & \textbf{B} & \textbf{C} & \textbf{D} & \textbf{E}\\\hline
''B'' & NULL & NULL & ''bdf'' & 1\\\hline
''B'' & 1500 & NULL & ''c'' & 1\\\hline
\end{tabular}
\end{center}

 \item right outer join
\begin{center}
\begin{tabular}{|c|c|c|c|c|}\hline
\textbf{A} & \textbf{B} & \textbf{C} & \textbf{D} & \textbf{E}\\\hline
''B'' & NULL & NULL & ''bdf'' & 1\\\hline
''B'' & 1500 & NULL & ''c'' & 1\\\hline
''C'' & NULL & NULL & NULL & 2\\\hline
''C'' & NULL & NULL & NULL & 3\\\hline
\end{tabular}
\end{center}
\vspace*{2cm}
 \item full outer join
\begin{center}
\begin{tabular}{|c|c|c|c|c|}\hline
\textbf{A} & \textbf{B} & \textbf{C} & \textbf{D} & \textbf{E}\\\hline
''A'' & 1000 & 3 & '''' & NULL\\\hline
''A'' & 700 & NULL & ''agh'' & NULL\\\hline
''A'' & NULL & 0 & ''abcdf'' & NULL\\\hline
''A'' & 1000 & 4 & NULL & NULL\\\hline
''B'' & NULL & NULL & ''bdf'' & 1\\\hline
''B'' & 1500 & NULL & ''c'' & 1\\\hline
NULL & 1000 & 8 & '''' & NULL\\\hline
NULL & 700 & 12 & NULL & NULL\\\hline
''C'' & NULL & NULL & NULL & 2\\\hline
''C'' & NULL & NULL & NULL & 3\\\hline
\end{tabular}
\end{center}
\end{itemize}

\section*{Aufgabe 2}
\lstset{frame=single}
Wir nehmen an, dass $r.C$ $n$ und $s.C$ $m$ Elemente hat.\\
Alle Algorithmen machen Gebrauch folgender Hilfsmethode:

\begin{lstlisting}[caption=Unsortierte Listen]{Name}
combine(tubel r, tupel s)
t = new tupel
n = size(r)
m = size(s)

for i=1 to n
  add r[i] to t

for j=2 to m
  add s[j] to t

return t
\end{lstlisting}

\begin{itemize}
 \item Bei beidseits unsortierten Listen muss f�r jedes Element in $s.C$ das passende Schl�sselelement in $r.C$ gefunden werden, und dies kann nur beim Durchgehen durch alle Elemente erreicht werden. Dies ergibt eine Zeitkomplexit�t von $T(n) = \Theta(n\cdot m)$. Nehmen wir an, dass $n$ und $m$ gleichschnell wachsen, so k�nnte man die Zeitkomplexit�t auf $\Theta(n^2)$ vereinfachen.
\begin{lstlisting}[caption=Unsortierte Listen]{Name}
naturalJoinUnsorted(list r, list s)
l = new list
n = size(r)
m = size(s)

for i=1 to n
  for j=1 to m
    if r[i] = s[j]
      add combine(r[i],s[j]) to l
      next i

return l
\end{lstlisting}

 \item Bei sortiertem $r.C$ muss f�r jedes Element in $s.C$ das passende Schl�sselelement in $r.C$ gefunden werden, und dies kann nur erreicht werden, indem man mit einer Suche in $r.C$ den (falls existierenden) richtigen Schl�ssel sucht, welches eine Zeit von $\Theta(log(n))$ braucht\footnote{Z.B. durch halbieren, pr�fen des mittleren Elements ob gr�sser oder kleiner des gesuchten, und das ganze zu wiederholen.}. Dies ergibt eine Zeitkomplexit�t von $T(n) = \Theta(n\cdot log(m))$. Nehmen wir an, dass $n$ und $m$ gleichschnell wachsen, so k�nnte man die Zeitkomplexit�t auf $\Theta(n\cdot log(n))$ vereinfachen.
\begin{lstlisting}[caption=r.C ist sortiert]{Name}
naturalJoinHalfSorted(list r, list s)
l = new list
n = size(r)
m = size(s)

for i=1 to n
  logSearch(r[i] in s.C)
    add combine(r[i],s[j]) to l

return l
\end{lstlisting}

 \item Bei sortiertem $r.C$ und sortiertem $s.C$ muss das passende Schl�sselelement in $r.C$ gefunden werden, und dies
kann sehr einfach erreicht werden, indem man f�r jedes Element von $s.C$ ($\Theta(n)$ wie in den vorigen) das
n�chsth�here Element in $r.C$ untersucht (welches in konstanter Zeit ($\Theta(1)$) m�glich ist\footnote{Um genau zu
sein, betr�gt die Laufzeit der inneren Schleife auch $\Theta(m)$, weil \texttt{key} durch jedes Element
l�uft (von 0 bis m). Daraus folgt eine gesamte Laufzeit von $\Theta(n\cdot m)$. Bei gleichschnellem Wachsen von $n$
und $m$ ergibt dies $\Theta(2n)$, welches eine asymptotische Laufzeit von $\Theta(n)$ ergibt.}). Dies ergibt insgesamt
eine Zeitkomplexit�t von $T(n) = \Theta(n)$.
\begin{lstlisting}[caption=Beidseits sortiert]{Name}
naturalJoinSorted(list r, list s)
l = new list
n = size(r)
m = size(s)
int key=0

for i=1 to n
  do
    if r[i] = s[key]
      add combine(r[i],s[j]) to l
    key++
    if r[i] < s[key]
      key-1
      next i
  while (r[i] < s[key])

return l
\end{lstlisting}

\end{itemize}

\section*{Aufgabe 3}
\begin{itemize}
 \item $\sigma_{A>10}(r-t) = \sigma_{A>10}(r) - \sigma_{A>10}(t)$. Es bleiben in beiden F�llen nur die $x,y,z$ �brig,
welche nur in $r$ enthalten sind und f�r welche $x>10$ gilt. Folgend ein Beispiel.
{%
\newcommand{\mc}[3]{\multicolumn{#1}{#2}{#3}}
\begin{center}
\begin{tabular}{ccccccc}
\textbf{} & \textbf{r} & \textbf{} & \textbf{} & \textbf{} & \textbf{t} & \textbf{}\\\cline{1-3}\cline{5-7}
\mc{1}{|c|}{\textbf{A}} & \mc{1}{c|}{\textbf{B}} & \mc{1}{c|}{\textbf{C}} & \mc{1}{c|}{\textbf{}} &
\mc{1}{c|}{\textbf{A}} & \mc{1}{c|}{\textbf{B}} & \mc{1}{c|}{\textbf{C}}\\\cline{1-3}\cline{5-7}
\mc{1}{|c|}{1} & \mc{1}{c|}{2} & \mc{1}{c|}{3} & \mc{1}{c|}{} & \mc{1}{c|}{1} & \mc{1}{c|}{2} &
\mc{1}{c|}{3}\\\cline{1-3}\cline{5-7}
\mc{1}{|c|}{7} & \mc{1}{c|}{9} & \mc{1}{c|}{2} & \mc{1}{c|}{} & \mc{1}{c|}{7} & \mc{1}{c|}{8} &
\mc{1}{c|}{5}\\\cline{1-3}\cline{5-7}
\mc{1}{|c|}{12} & \mc{1}{c|}{7} & \mc{1}{c|}{8} & \mc{1}{c|}{} & \mc{1}{c|}{12} & \mc{1}{c|}{7} &
\mc{1}{c|}{8}\\\cline{1-3}\cline{5-7}
\mc{1}{|c|}{13} & \mc{1}{c|}{1} & \mc{1}{c|}{2} & \mc{1}{c|}{} & \mc{1}{c|}{11} & \mc{1}{c|}{1} &
\mc{1}{c|}{2}\\\cline{1-3}\cline{5-7}
\end{tabular}
\end{center}
}%

\begin{center}
\begin{tabular}{|c|c|c|}\hline
\textbf{A} & \textbf{B} & \textbf{C}\\\hline
13 & 1 & 2\\\hline
\end{tabular}
\end{center}

 \item $r \bowtie (\sigma_{B=''X'' \vee C=''Z''}(s))$: Von (s) werden nur die Zeilen ausgew�hlt f�r die gilt $B=''X''
\vee C=''Z''$\\
Mit $\bowtie$ werden nur die Zeilen beachtet f�r die das gleiche gilt.

$\sigma_{B=''X'' \vee C=''Z''}(r \bowtie s)$: Mit $r \bowtie s$ werden zwar alle Zeilen beachtet for die gilt
$B(r)=B(s) \vee C(r)=C(s)$, aber nur die ausgew�hlt wo $B=''X''$ und $C=''Z''$. $\Rightarrow$ Damit sind die Ausdr�cke
�quivalent.

 \item Angenommen $z\in B(r) \vee z\notin B(s)$. $r \div \pi_{B}(s)$ enth�lt nur die Spalten A und C von r. \\
$r \div \pi_{B}(s) \times \pi_{B}(s)$ enth�lt nur die Spalten A(r), C(r) und B(s). Dass heisst, alle Elemente von B(r)
die nicht Elemente von B(s) sind k�nnen nicht im Kreuzprodukt enthalten sein. D.h, auf der linken Seite w�re ein $z$
vorhanden, aber auf der rechten nicht.

\end{itemize}

\end{document}