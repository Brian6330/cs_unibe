\documentclass[a4paper,10pt,headlines=3.2]{scrartcl}
\usepackage{graphicx}           %Bilder

%\usepackage[T1]{fontenc}        %Umlaute
%\usepackage[latin1]{inputenc}   %Windows
%\usepackage[utf8x]{inputenc}	%Linux
\usepackage{ucs}

\usepackage[ngerman]{babel}     %Deutsche Sprache
\usepackage{amsmath}            %Math. Zeichen
\usepackage{pifont}             %Skalierbare Schriftart
\usepackage{array}
\usepackage{epsfig}             %Erweiterte Grafiken
\usepackage{makeidx}            %Stichwortverzeichnis
\usepackage[pdftex]{color} 

\newcommand{\changefont}[3]{
\fontfamily{#1} \fontseries{#2} \fontshape{#3} \selectfont}

\makeindex

\usepackage[automark]{scrpage2}
\usepackage[nosectionbib]{apacite}               %Zitieren

%\usepackage[colorlinks]{hyperref}%Hyperlinks

\usepackage{lmodern}
\usepackage{scrpage2}           %KOMA-Script
\usepackage{tipa}
\usepackage{qtree}

\usepackage{remreset}			%Fussnoten global
\makeatletter
\@removefromreset{footnote}{chapter}
\makeatother 

\setcounter{tocdepth}{3}

%Kopfzeilen
\pagestyle{scrheadings}         %Seitenstil scrheadings verwenden

%\setlength{\textheight}{24cm}
%\setlength{\textwidth}{16cm}
%\setlength{\topmargin}{-2cm}
%\setlength{\oddsidemargin}{0cm}

% Groesse des Textbereiches in der Seite
\setlength{\textwidth}{16cm}
\setlength{\textheight}{22cm}
% Kopf- und Fusszeile, Hoehe und Abstand vom Text
\setlength{\headheight}{15pt}
\setlength{\headsep}{0.8cm}
% Linker Seiteneinzug
\setlength{\oddsidemargin}{2.5cm} \addtolength{\oddsidemargin}{-1in}
\setlength{\evensidemargin}{2.5cm} \addtolength{\evensidemargin}{-1in}
% Andere Groessen ausrechnen (vertikal zentrieren)
\setlength{\footskip}{\headsep}
\addtolength{\footskip}{\headheight}
\setlength{\topmargin}{\paperheight}
\addtolength{\topmargin}{-\textheight}
\addtolength{\topmargin}{-\headheight}
\addtolength{\topmargin}{-\headsep}
\addtolength{\topmargin}{-\footskip}
\addtolength{\topmargin}{-2in}
\addtolength{\topmargin}{-0.5\topmargin}

%Abstand zur�cksetzen
\setlength{\headheight}{20pt}

\usepackage{listings} 
\lstset{numbers=left, numberstyle=\tiny, numbersep=5pt} \lstset{language=Java} 
\changefont{cmss}{m}{n}

\clearscrheadfoot
%\renewcommand{\headheight}{40pt} 
\ihead[]{Datenbanken \\Fr�hlingssemester 2011 \\Institut f�r angewandte
Mathematik} % - linke Kopfzeile 
\ohead[asdasd]{�bung 3, Abgabe 15. M�rz 2011 \\Adrianus Kleemans
[07-111-693]\\Pinar Kayalar [10-123-453]} % - linke Kopfzeile 
\setheadsepline{.4pt} %Separate Linie im Kopf
\cfoot[\pagemark]{\pagemark} %- mittlere Fusszeile 


\begin{document}
\section*{Aufgabe 1}
\begin{itemize}
 \item[1.] Zur Vereinfachung f�hren wir f�r die vorhandenen Attribute Aliase
ein: \textit{A=TelefonNr, B=Talort, C=Skigebiet, D=Lift, E=Kapazit�t}.\\
Folgende candidate keys ergeben sich aus den eingetragenen Werten: [AD], [AE],
[BD], [BE], [ABE], [ABD], [ACD], [ACE], [ADE], [BCD], [BCE], [BDE], [ABCD], [ABCE], [ABDE],
[ACDE], [BCDE], [ABCDE].
 \item[2.] Am besten wird ein Prim�rschl�ssel gew�hlt, bei welchem auch
zuk�nftige Tupel keine redundanten Werte aufweisen. Dies w�re z.B. bei [BD] der
Fall, also \textit{Talort} und \textit{Lift}, denn es ist anzunehmen, dass es
keinen gleich benannten Lift in einem Ort (\textit{Talort}) gibt.
\end{itemize}

\section*{Aufgabe 2}
\begin{itemize}
 \item $\pi_{a1}(r)$\\
\begin{center}
\begin{tabular}{|c|}\hline
\textbf{a1}\\\hline
a\\\hline
b\\\hline
c\\\hline
\end{tabular}
\end{center}
 \item $\sigma_{a1=``b``}(r)$\\
\begin{center}
\begin{tabular}{|c|c|}\hline
\textbf{a1} & \textbf{a2}\\\hline
b & d\\\hline
b & e\\\hline
\end{tabular}
\end{center}

 \item $r\times s$\\
\begin{center}
\begin{tabular}{|c|c|c|c|}\hline
\textbf{a1} & \textbf{r.a2} & \textbf{s.a2} & \textbf{a3}\\\hline
a & d & d & g\\\hline
a & d & e & h\\\hline
b & d & d & g\\\hline
b & d & e & h\\\hline
b & e & d & g\\\hline
b & e & e & h\\\hline
c & f & d & g\\\hline
c & f & e & h\\\hline
\end{tabular}
\end{center}


 \item $\sigma_{r.a2=s.a2}(r\times s)$\\
\begin{center}
\begin{tabular}{|c|c|c|c|}\hline
\textbf{a1} & \textbf{r.a2} & \textbf{s.a2} & \textbf{a3}\\\hline
a & d & d & g\\\hline
b & d & d & g\\\hline
b & e & e & h\\\hline
\end{tabular}
\end{center}

 \item $\pi_{a1}(r) - \pi_{a1}(\sigma_{a2=``d``}(r))$\\
\begin{center}
\begin{tabular}{|c|}\hline
\textbf{a1}\\\hline
c\\\hline
\end{tabular}
\end{center}

\end{itemize}

\section*{Aufgabe 3}
\begin{itemize}
 \item[1] $\pi_{person-name}(\sigma_{company-name=``FBC``}(works))$
 \item[2] $\pi_{employee.person-name,
employee.city}(\sigma_{works.company-name=``FBC``}(employee\bowtie works))$
 \item[3] $\pi_{employee.person-name, employee.street,
employee.city}(\sigma_{works.company-name=``FBC``,works.salary>100000}
(employee\bowtie works))$
 \item[4]
$\pi_{employee.person-name}(\sigma_{employee.city=company.city}((employee\bowtie
works)\bowtie company))$
\end{itemize}

\section*{Aufgabe 4}
\lstset{frame=single}
\begin{itemize}
 \item[1] Eine Menge kann im Unterschied zur Liste keine gleichen Elemente enthalten, denn jedes Element muss ``wohlunterschieden`` sein. Eine Liste hat von ihrem Aufbau her immer auch schon eine implizierte Reihenfolge Ihrer Elemente.
Zudem kann eine Menge endlich oder unendlich viele Elemente enthalten, wohingegen eine Liste nur eine endliche Anzahl von Elementen aufnehmen kann.
\end{itemize}

\begin{lstlisting}[caption=Aufgabe 4.2]{Name}
boolean equals(list l1, list l2)

int m,n
for m=0 to list1.length
    for n=0 to list2.length
	if list1[m] == list2[n] 
	    break
	else if n == list2.length-1
	    return false
return true
\end{lstlisting}

\end{document}
